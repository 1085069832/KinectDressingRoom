\documentclass[a4paper]{article}

\usepackage{amsmath}
\usepackage{amsfonts}
\usepackage{amssymb}
\usepackage{url}

\title{Kinect Virtual Dressing Room}

\author{Fedde Burgers \\ 5705509 \\ \texttt{f.j.b.burgers@uva.nl} \and Morris Franken \\ 6151825 \\ \texttt{morrisef@science.uva.nl} \and Carsten van Weelden \\ 0518824 \\ \texttt{cweelden@science.uva.nl}}

\begin{document}

\maketitle

\begin{abstract}

\end{abstract}

\par\noindent {\small{\em Keywords\/}: augmented reality, virtual clothing, user tracking, kinect}

\section{Introduction}

The gaming industry has recently introduced the Kinect sensor which has interesting academic applications. It allows one to generate a depth image alongside a RGB camera image and comes with tools that provide human pose detection and tracking. These abilities are used by the gaming industry to allow users to control games using body movement. It can also be used to create an immersive virtual reality presence for the user such as in the Kinect Superman project\footnote{\url{https://github.com/kinectsuperman/Kinect-Superman}} and also to create augmented reality applications in which virtual objects seem to interact with the user and his environment.

In this project we use the Kinect sensor to create an augmented reality dressing room in which the user can try on virtual clothes. The pose of the user is tracked to allow the clothing to move with the user and the depth image is used to create an avatar of the user that approximates the user's body shape. Next, cloth simulation is applied to the virtual clothing to make it move and fold realistically based on the user's movements. The depth image from the sensor is used to compute the girth of the user's body at several places, which can be used to adapt the user avatar, adapt the clothing to the user's body shape, and recommend clothing sizes to the user. The RGB image is used as a background over which the clothing is projected and is displayed on the user avatar when it occludes parts of the clothing. The user is segmented from the background and the intensity of this part of the image is calculated to adapt the lighting of the virtual clothes. This makes the clothing appear as if it is in the same room as the user by reacting to bright and shaded of the environment.

\subsection{Project goals}

The focus of this project is to create as realistic an augmented reality dressing room as possible. For this real-time, accurate, tracking of the user pose is needed as well as realistic virtual clothing. For the pose tracking the Kinect sensor is used which gives more complete and accurate tracking of the user pose than the marker based or image-feature based tracking which is traditionally used in augmented reality applications.

\section{Related work}

Evolution of the virtual dressing room idea: avatar with static clothes, AR with markers, full body with static clothes etc. Dimensions along which to organize: user tracking ability (none, markers, full skeleton tracking), clothing simulation (2d images, static clothes from multiple angles, movable clothes, full physics simulation),

The idea of trying on virtual clothes is not new. With the growing use of digital cameras and more and more households possessing a webcam, it was possible to try on clothes by overlaying an image of the clothing over the camera image. Like every other technique the virtual dressing room evolved from very simple to more inventive solutions. The evolutionairy differences in these solutions can be largely reduced to two dimensions: the alignment of the user and the clothing and the authenticity of the clothing.

\subsection{Alignment of user and clothing}

In the first virtual dressing rooms there was no tracking of the user at all. In this very primitive form of augmented reality only an image of the clothing was overlayed over the camera image on a fixed position. In order to get the visual experience of wearing the clothing, the user had to align his body to the clothing image hisself. 

% Insert screenshot

A more appropriately manner of alignment would be to adjust the position, rotation and scale of the clothing to the user. The use of a marker made it possible to receive some 3D information out of the RGB image of a normal webcam. Position, rotation and scale were adjusted by moving the marker.

% Insert screenshot

The introduction of the Kinect gave relatively easy and cheap access to a depth camera. And with provided middelware such as the OpenNI framework the user's pose is trackable in a quite accurate way. FaceCake implemented a virtual dressing room using the Kinect and aligns the image of the clothing with the user's body using the pose tracking. This solution is currently the state of the art dressing room.
The FaceCake virtual dressing room has some limitations though. In their application only upper body clothing with short sleeves is included and some accessories like bags. The legs are completely ignored and the arms are only used for the accessories.

% Insert FaceCake screenshot

\subsection{Authenticity of clothing}

The goal of a virtual dressing room is to give a realistic visual experience of trying on clothes. Beside the alignment of the clothing, the authenticity is an important aspect in providing this experience. In the first versions the clothes were just static 2D images. It was only possible to see how the clothes looked from the front. In later dressing rooms such as the one created by FaceCake multiple 2D images of the clothing from different angles provided a more realistic experience, as it was possible to turn around and have a look from different angles. The clothing is still static and there is completely no interaction with the clothes besides changing location, rotation and scale.

In our approach a great improvement over the state of the art virtual dressing rooms could be made on both dimensions. In the first place the Kinect with the pose tracking gives a full skeleton of the user. Although they might be harder to track, we think that a full dressing room experience should have the option to try on pants and longsleeves too. And as our approach makes an avatar of the user and puts that avatar clothes on, it is possible to use 3D models of the clothing. Also the interaction possibly could be more realistic; the 3D modeled clothing with physics folds like real clothing and the user's avatar can have full interaction with the clothing.

\section{Implementation}

\subsection{User tracking}

\subsection{Virtual clothes}

clothing, lightness

\subsection{Size estimation}

\subsection{Application overview}

Walkthrough, gui, mooie plaatjes

\section{Discussion}

Interaction with cloth using collision detection, mislukte interactive cloth approach

Pixel shaders vs user body (zou als groot voordeel hebben dat je ook andere objecten (bv. tafel waar je achter staat) de kleding zouden occlude-en

Evaluatie van size estimation moet grotere dataset hebben

Kleding zou meer realistisch zijn met hulp van 3d artists (textures, pasvorm)

\section{Conclusion}

Evaluation of project goals in section 1
\end{document}
